\documentclass[12pt]{article}
\usepackage{amsmath}
\usepackage{enumerate}
\usepackage[margin=1in]{geometry}
\usepackage{booktabs, multicol, multirow}
\usepackage[capposition=top]{floatrow}
\usepackage{graphicx}
\usepackage{csquotes}
\usepackage{lscape}
\usepackage{array}
\usepackage{caption}
\usepackage{subcaption}
\usepackage{dcolumn}
\usepackage{varwidth}
\usepackage{endnotes}
\usepackage{indentfirst}
\usepackage[hyphens]{url}
%\usepackage{breakurl} 
\usepackage[breaklinks]{hyperref}
\usepackage[table]{xcolor}
\usepackage{setspace}
\usepackage{alltt}
\usepackage{tabularx}
\usepackage{verbatim}

\let\endnote=\footnote
\urlstyle{same}

\newcolumntype{L}[1]{>{\raggedright\let\newline\\\arraybackslash\hspace{0pt}}m{#1}}
\newcolumntype{C}[1]{>{\centering\let\newline\\\arraybackslash\hspace{0pt}}m{#1}}
\newcolumntype{M}[1]{>{\begin{varwidth}[t]{#1}}l<{\end{varwidth}}}
\newcolumntype{X}{>{\centering\arraybackslash}m{3.4cm}}
\newcolumntype{Y}{>{\centering\arraybackslash}m{4.2cm}}

\newcommand{\specialcell}[2][c]{\begin{tabular}[#1]{@{}l@{}}#2\end{tabular}}


\begin{document}

\begin{table}[h!]
\small
\caption*{\bf Table 3: Assessments of RttT Influence by State Legislators}
{
\def\sym#1{\ifmmode^{#1}\else\(^{#1}\)\fi}
\begin{tabular}{l*{4}{C{2.1cm}C{2.1cm}C{2.1cm}C{2.1cm}}}
\toprule
            &All states (\%)&RttT winning states (\%)&RttT applicants (\%)&RttT non-applicants (\%)\\
\midrule
Massive influence&        5.69&        8.51&        4.92&        4.35\\
Big influence&       26.24&       52.13&       20.08&        8.70\\
Minor influence&       49.01&       37.23&       53.79&       45.65\\
No influence at all&       19.06&        2.13&       21.21&       41.30\\
Total       &      100.00&      100.00&      100.00&      100.00\\
\midrule
Number of respondents&         404&          94&         264&          46\\
\bottomrule
\multicolumn{5}{l}{\footnotesize \begin{minipage}{5.5in} \vspace{3mm} \footnotesize                                  This table reports responses from an online survey conducted in the spring of 2014 in which state legislators were                                  asked the following question: \enquote{In an effort to encourage state governments to pass elements of his education agenda,                                  President Obama in 2010 launched a series of competitions known as Race to the Top. In these competitions, states had                                  a chance of winning federal monies in exchange for their commitments to enact a series of specific education policies                                  that were supported by the federal Department of Education. We're wondering what impact, if any, these initiatives have                                  had on education policymaking in your state.  Have they had a massive impact, a big impact, a minor impact, or no impact                                  at all?} A two-tailed t-test finds the responses of each paired combination of winners, losers, and non-applicants to be                                 statistically significantly different from one another at the p$<$0.01 level.                                  \end{minipage}}\\
\end{tabular}
}

\end{table}
\clearpage

\begin{landscape}
\begin{table}[h!]
\footnotesize
\caption*{\bf Table 4: RttT Policy Enactment among Winners and Losers}
{
\def\sym#1{\ifmmode^{#1}\else\(^{#1}\)\fi}
\begin{tabular}{l*{8}{C{1.6cm}C{1.6cm}C{1.6cm}C{1.6cm}C{1.6cm}C{1.6cm}C{1.6cm}C{1.6cm}}}
\toprule
                    &\multicolumn{1}{c}{(1a)}&\multicolumn{1}{c}{(1b)}&\multicolumn{1}{c}{(2a)}&\multicolumn{1}{c}{(2b)}&\multicolumn{1}{c}{(3a)}&\multicolumn{1}{c}{(3b)}&\multicolumn{1}{c}{(4a)}&\multicolumn{1}{c}{(4b)}\\
                    &RttT Policies   &Control Policies   &RttT Policies   &Control Policies   &RttT Policies   &Control Policies   &RttT Policies   &Control Policies   \\
\midrule
Won RttT (up to time t)&       0.408***&       0.289** &       0.406***&       0.283** &       0.519***&       0.064   &       0.592** &      -0.287***\\
                    &     (0.032)   &     (0.114)   &     (0.032)   &     (0.114)   &     (0.001)   &     (0.039)   &     (0.247)   &     (0.096)   \\
Applied and lost RttT (up to time t)&       0.207***&       0.123   &       0.205***&       0.120   &       0.549***&       0.044***&       0.620** &      -0.305***\\
                    &     (0.033)   &     (0.099)   &     (0.033)   &     (0.099)   &     (0.046)   &     (0.004)   &     (0.235)   &     (0.096)   \\
State education revenue per capita&               &               &               &               &               &               &       0.109   &       0.003   \\
                    &               &               &               &               &               &               &     (0.151)   &     (0.184)   \\
Democratic governor &               &               &               &               &               &               &      -0.019   &      -0.072*  \\
                    &               &               &               &               &               &               &     (0.032)   &     (0.040)   \\
Democratic majority, both chambers&               &               &               &               &               &               &       0.039   &       0.017   \\
                    &               &               &               &               &               &               &     (0.027)   &     (0.035)   \\
Constant            &       0.160***&       0.137   &       0.052   &       0.012   &       0.076   &       0.180***&      -0.150   &       0.178   \\
                    &     (0.026)   &     (0.087)   &     (0.063)   &     (0.089)   &     (0.058)   &     (0.041)   &     (0.314)   &     (0.376)   \\
Year fixed effects  &         Yes   &         Yes   &         Yes   &         Yes   &         Yes   &         Yes   &         Yes   &         Yes   \\
Policy fixed effects &          No   &          No   &         Yes   &         Yes   &         Yes   &         Yes   &         Yes   &         Yes   \\
State fixed effects &          No   &          No   &          No   &          No   &         Yes   &         Yes   &         Yes   &         Yes   \\
\midrule
$ R ^ {2} $         &       0.091   &       0.047   &       0.353   &       0.142   &       0.424   &       0.499   &       0.424   &       0.501   \\
N                   &        4762   &         689   &        4762   &         689   &        4762   &         689   &        4762   &         689   \\
\bottomrule
\end{tabular}
}

\begin{minipage}{\textwidth} 
{\footnotesize Per capita income and state revenue and expenditure variables are in thousands of dollars. Elementary-secondary school enrollment and state population are in thousands. Standard errors are clustered at the state level. \\
 *** p$<$0.01, ** p$<$0.05, * p$<$0.10.}
\end{minipage}
\end{table}
\end{landscape}
\clearpage

\begin{table}[h!]
\footnotesize
\caption*{\bf Table 5: Linking RttT Policy Enactments to Application Promises}
{
\def\sym#1{\ifmmode^{#1}\else\(^{#1}\)\fi}
\begin{tabular}{l*{2}{C{1.6cm}C{1.6cm}C{1.6cm}C{1.6cm}C{1.6cm}C{1.6cm}}}
\toprule
                    &\multicolumn{1}{c}{(1)}   &\multicolumn{1}{c}{(2)}   \\
\midrule
Won RttT (up to time t)&      -0.011   &      -0.011   \\
                    &     (0.078)   &     (0.077)   \\
Promise * won       &       0.200***&       0.201***\\
                    &     (0.063)   &     (0.062)   \\
Promise * applied and lost&       0.216***&       0.217***\\
                    &     (0.036)   &     (0.036)   \\
State education revenue per capita&               &       0.117   \\
                    &               &     (0.209)   \\
Democratic governor &               &      -0.016   \\
                    &               &     (0.035)   \\
Democratic majority, both chambers&               &       0.042   \\
                    &               &     (0.027)   \\
Constant            &      -0.002   &       0.019   \\
                    &     (0.058)   &     (0.308)   \\
Year fixed effects  &         Yes   &         Yes   \\
Policy fixed effects &         Yes   &         Yes   \\
State fixed effects &         Yes   &         Yes   \\
\midrule
$ R ^ {2} $         &       0.446   &       0.446   \\
N                   &        4384   &        4384   \\
\bottomrule
\end{tabular}
}

\vspace{2mm}
\begin{minipage}{.85\textwidth} 
{\footnotesize Per capita income and state revenue and expenditure variables are in thousands of dollars. Elementary-secondary school enrollment and state population are in thousands. Standard errors are clustered at the state level. \\
 *** p$<$0.01, ** p$<$0.05, * p$<$0.10.}
\end{minipage}
\end{table}
\clearpage

\begin{table}[h!]
\footnotesize
\captionsetup{justification=centering}
\caption*{\bf Table 6: Average Treatment Effect on the Treated \\ Exact Matching on Year and Policy Domain, Nearest Neighbor Matching on Section Score}
Comparison 1: 2010-11, treated observations include Phase 1 and 2 winners and untreated observations include all others \\
{
\def\sym#1{\ifmmode^{#1}\else\(^{#1}\)\fi}
\begin{tabular}{l*{6}{C{2.2cm}C{2.2cm}C{2.2cm}}}
\toprule
&\multicolumn{1}{c}{(1)} &\multicolumn{1}{c}{(2)} &\multicolumn{1}{c}{(3)} \\
& Full data & Cal=0.1 SD & Cal=0.05 SD \\
\midrule
ATT &0.120*** & 0.115*** & 0.098** \\ 
& (0.032) & (0.037) & (0.038) \\ 
\midrule
N & 936 & 676 & 652 \\
\bottomrule
\end{tabular}
}
\vspace{5mm}
Comparison 2: 2012-13, treated observations include Phase 3 winners and untreated observations include applicants that never won \\
{
\def\sym#1{\ifmmode^{#1}\else\(^{#1}\)\fi}
\begin{tabular}{l*{6}{C{2.2cm}C{2.2cm}C{2.2cm}}}
\toprule
&\multicolumn{1}{c}{(1)} &\multicolumn{1}{c}{(2)} &\multicolumn{1}{c}{(3)} \\
& Full data & Cal=0.1 SD & Cal=0.05 SD \\
\midrule
ATT &0.093*** & 0.156*** & 0.154*** \\ 
& (0.033) & (0.036) & (0.037) \\ 
\midrule
N & 776 & 628 & 596 \\
\bottomrule
\end{tabular}
}
\vspace{5mm}
Comparison 3: 2012-13, treated observations include Phase 1 and 2 winners and untreated observations include all others (including Phase 3 winners) \\
{
\def\sym#1{\ifmmode^{#1}\else\(^{#1}\)\fi}
\begin{tabular}{l*{6}{C{2.2cm}C{2.2cm}C{2.2cm}}}
\toprule
&\multicolumn{1}{c}{(1)} &\multicolumn{1}{c}{(2)} &\multicolumn{1}{c}{(3)} \\
& Full data & Cal=0.1 SD & Cal=0.05 SD \\
\midrule
ATT &0.118*** & 0.094*** & 0.073** \\ 
& (0.025) & (0.029) & (0.030) \\ 
\midrule
N & 1302 & 914 & 882 \\
\bottomrule
\end{tabular}
}
\vspace{2mm}
\begin{minipage}{\textwidth} 
{\footnotesize Standard errors are clustered at the state level. \\
 *** p$<$0.01, ** p$<$0.05, * p$<$0.10.}
\end{minipage}
\end{table}
\clearpage


\begin{table}[h!]
\footnotesize
\captionsetup{justification=centering}
\caption*{\bf Table 7: Effect of Adoption of Same Policy in Proximate States}
{
\def\sym#1{\ifmmode^{#1}\else\(^{#1}\)\fi}
\begin{tabular}{l*{4}{C{1.8cm}C{1.8cm}C{1.8cm}C{1.8cm}C{1.8cm}C{1.8cm}}}
\toprule
                    &\multicolumn{1}{c}{(1)}   &\multicolumn{1}{c}{(2)}   &\multicolumn{1}{c}{(3)}   &\multicolumn{1}{c}{(4)}   \\
\midrule
Won RttT (up to time t)&       0.314***&       0.335***&       0.291***&       0.326***\\
                    &     (0.041)   &     (0.048)   &     (0.080)   &     (0.073)   \\
Applied and lost RttT (up to time t)&       0.188***&       0.208***&       0.157** &       0.190***\\
                    &     (0.039)   &     (0.044)   &     (0.077)   &     (0.070)   \\
Same policy in similar states&       0.258***&       0.258***&               &               \\
                    &     (0.013)   &     (0.013)   &               &               \\
Same policy in neighboring states&               &               &       0.145***&       0.146***\\
                    &               &               &     (0.017)   &     (0.017)   \\
State education revenue per capita&               &       0.019   &               &       0.031   \\
                    &               &     (0.019)   &               &     (0.024)   \\
Democratic governor &               &       0.006   &               &       0.004   \\
                    &               &     (0.016)   &               &     (0.016)   \\
Democratic majority, both chambers&               &      -0.006   &               &      -0.002   \\
                    &               &     (0.019)   &               &     (0.022)   \\
Constant            &       0.199***&       0.163***&       0.185***&       0.126*  \\
                    &     (0.035)   &     (0.052)   &     (0.048)   &     (0.073)   \\
Year fixed effects  &         Yes   &         Yes   &         Yes   &         Yes   \\
Policy fixed effects &         Yes   &         Yes   &         Yes   &         Yes   \\
State fixed effects &         Yes   &         Yes   &         Yes   &         Yes   \\
\midrule
$ R ^ {2} $         &       0.520   &       0.521   &       0.487   &       0.488   \\
N                   &       12101   &       12101   &       12101   &       12101   \\
\bottomrule
\end{tabular}
}

\vspace{2mm}
\begin{minipage}{.95\textwidth} 
{\footnotesize Same policy in similar and neighboring states, state population, percent Black and Hispanic, elementary-secondary school enrollment, per capita income, and state education revenue and expenditure variables are standardized, so that the interpretation of their coefficients is the change in the outcome associated with a one standard deviation increase in the explanatory variable. Standard errors are clustered at the state level. \\
 *** p$<$0.01, ** p$<$0.05, * p$<$0.10.}
\end{minipage}
\end{table}
\clearpage
\end{document}
