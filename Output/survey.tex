{
\def\sym#1{\ifmmode^{#1}\else\(^{#1}\)\fi}
\begin{tabular}{l*{4}{C{2.1cm}C{2.1cm}C{2.1cm}C{2.1cm}}}
\toprule
            &All states (\%)&RttT winning states (\%)&RttT applicants (\%)&RttT non-applicants (\%)\\
\midrule
Massive influence&        5.69&        8.51&        4.92&        4.35\\
Big influence&       26.24&       52.13&       20.08&        8.70\\
Minor influence&       49.01&       37.23&       53.79&       45.65\\
No influence at all&       19.06&        2.13&       21.21&       41.30\\
Total       &      100.00&      100.00&      100.00&      100.00\\
\midrule
Number of respondents&         404&          94&         264&          46\\
\bottomrule
\multicolumn{5}{l}{\footnotesize \begin{minipage}{5.5in} \vspace{3mm} \footnotesize                                  This table reports responses from an online survey conducted in the spring of 2014 in which state legislators were                                  asked the following question: \enquote{In an effort to encourage state governments to pass elements of his education agenda,                                  President Obama in 2010 launched a series of competitions known as Race to the Top. In these competitions, states had                                  a chance of winning federal monies in exchange for their commitments to enact a series of specific education policies                                  that were supported by the federal Department of Education. We're wondering what impact, if any, these initiatives have                                  had on education policymaking in your state.  Have they had a massive impact, a big impact, a minor impact, or no impact                                  at all?} A two-tailed t-test finds the responses of each paired combination of winners, losers, and non-applicants to be                                 statistically significantly different from one another at the p$<$0.01 level.                                  \end{minipage}}\\
\end{tabular}
}
